\documentclass{article}

\usepackage{lastpage}

\usepackage{helvet}
\renewcommand{\familydefault}{\sfdefault}

\usepackage[utf8]{inputenc}
\usepackage[ngerman]{babel}
\usepackage[T1]{fontenc}
\usepackage{hyperref}

\usepackage[a4paper, margin=1.2in]{geometry}
\usepackage{fancyhdr}

\usepackage{tabularx}

\usepackage{pbox}

\usepackage{spreadtab}
\usepackage{comment}

\usepackage{parskip}

\usepackage{paralist}
\usepackage{color}

\pagestyle{fancy}


%owncommands
\newcommand{\gal}{\texttt{Galaxy}\, }
\newcommand{\eg}{e. g.,\, }

\lhead{Midterm report\\
  „Deutsches Netzwerk für Bioinformatik-Infrastruktur - de.NBI“\\
  RNA bioinformatics center (RBC) - Leipzig}
\cfoot{}
\rfoot{\hrule%
  page \thepage\ / \pageref{LastPage}}


\excludecomment{comment}

\begin{document}

\section*{Projektleiter(in)}

{
\def\arraystretch{2.5}
\begin{tabular}[t]{p{7cm}@{}>{\bf}l}
Name: & 
\pbox[t]{7cm}{Prof.\\ 
Peter Stadler}\\
Förderkennzeichen: & 031A538B\\
Projketlaufzeit: & 01.03.2015 - 28.02.2018\\
Abteilung/AG: & 
\pbox[t]{7cm}{Fak. Mathematik und Informatik\\
  Bioinformatik/IZBI}\\
Institut/Universität: &
\pbox[t]{7cm}{Universität Leipzig\\
Härtelstrasse 16-18\\
04107 Leipzig}\\
\end{tabular}
}

\pagebreak

\section*{Summary: Overview on the activities of the individual project}
\label{sec:summary}
Leipzig is part of the de.NBI RNA bioinformatics center (RBC). 
The RBCs main objectives within de.NBI are as follows:

\begin{enumerate}
\item Tool interface descriptions
\item Galaxy integration
\item Virtualisation
\item Quality Management
\item Genome Annotation
\item Maintenance \& Optimisation
\item Coordination \& Training
\item Community \& User Support
\end{enumerate}

This tasks are tackled in an cooperative effort involving all three
RBC service centers.
The RBC-Leipzig contributed as follows:

Spearheaded by RBC-Freiburg, the \texttt{Galaxy} framework was
established as central workflow system within the RBC.  A major
objective therefore, is the integration of existing tools in the
Galaxy Tool Shed, making them accessible to researchers, enabling
reproducibility and long time availability. To this purpose, many of
the tools (co-)developed at Leipzig have been or are currently being
integrated into the Galaxy workflow system. An overview of this tools
can be found in section~\ref{sec:tools}. As most of the provided tools
are under constant development, this is a continuous process. For
integration into the galaxy framework, these tools are being made
available via the \texttt{BioConda} package manager, further
increasing visibility and usability.

Besides development, maintenance and \gal integration of tools,
another major topic that concerns the RBC-Leipzig is quality
management of RNA-seq and other RNA centric high-throughput sequencing
(HTS) data. Initial efforts were directed at an exhaustive survey of
current state-of-the-art sequencing technologies and established
quality control pipelines, with focus on non-standard applications and
their specific needs. The fast evolving field of RNA-HTS raised our
awareness that instead of formulating constraint and inflexible
standard operating procedures (SOPs) for a snapshot of technologies
and methods available at a certain timepoint, a more flexible,
adaptale approach has to be found. Especially recent advances in the
analysis of RNA interaction or epigenomics speed up this
development. To this end, we decided to implement a collection of
"best-practice" documents in favor if SOPs. The collection is hosted
on \texttt{Github}
(\href{https://github.com/galaxyproject/training-material}{https://github.com/galaxyproject/training-material}),
together with a series of how-to's and training material for \gal and
NGS analysis in general.  This guarantees that the material is
publicly available, and can be brought up-to-date by contributors in a
revision controlled, community driven and reviewed way, while
simultaneously visibility is increased. This allows to keep up with
developments in the field easier, as not only the RBC members, but
also scientist who use the tools we provide as well as other experts
can contribute their knowledge to the collection in an easy and
convenient way, while the community can review changes and stays in
control of what is integrated in the collection.

Consequently, adaption of analysis and quality control tools like \eg
\texttt{FastQC}, which are necessary in response to recent advances,
is an ongoing process. This also includes the development of new
tools, as \eg \texttt{Sierra platinum}, which is specifically designed
for multi-replicate peak calling and quality control of NGS data.



\begin{comment}
----- Jun 2015
In Workpackage 1.1 sollen innerhalb des ersten Jahres
Interfacebeschreibungssprachen evaluiert werden, um eine fundierte
Wahl einer standardisierten interface description language und eines
Workflowsystems zu treffen. Hier wird vor allem das Workflowsystems
Galaxy genutzt, das eine XML-basierte Toolinterface-Beschreibung
vorsieht. Gegenwärtig arbeiten wir an der Integration erster
bestehender Leipziger Tools (z.B. LocARNA) in Galaxy. Die endgültige
Integration ist für WP 1.2 vorgesehen (Monat 7-36)
%
In Workpackage 4.1 ist für die ersten 1.5 Jahre die
Entwicklung angepasster Qualitycontrolpipelines geplant. Hier sollen
zunächst, als Basis für die Adaption von standard Pipelines
(z.B. FastQ) die spezifischen nicht-standard Erfordernisse
RNA-bezogener Experimente analysiert werden.
%
Workpackages 6.1 und 6.2 dienen der Pflege und Anpassung bestehender
Tools. Die Identifikation der Benutzeranforderungen umfasst die
Analyse und Bearbeitung direkter Benutzeranfragen; aber auch eine
weitergehende koordinierte Analyse, um zu einheitlichen Interfaces und
Standards führen zu können. Workpackage 6.2 dient der konkreten Pflege
der bestehenden Software.  An diesen Punkten im Berichtszeitraum
kontinuierlich gearbeitet und führte zum Release neuer Toolversionen
(LocARNA 1.8.1 [http://www.bioinf.uni-freiburg.de/Software/LocARNA/],
MEA 0.3 [http://www.bioinf.uni-leipzig.de/Software/mea/]). Die
zentrumweite Koordination der Erfahrungen der einzelnen Partner ist
für das anstehende RBC-Treffen im Juli 2015 geplant. Bezüglich WP 8.1
wurde Support geleistet, z.B. im Design von Riboswitches
(Prof. Narberhaus, Prof. Dersch) und genomweitem Structureprobing
(Prof. Moerl); bzgl. 8.2 wurde die Dokumentation unserer Tools
weitergepflegt.
----- Dez 2015

Verschiedene Interfacebeschreibungssprachen wurden evaluiert und schliesslich durch die Wahl
des Workflowsystem Galaxy entschieden (Workpackage 1.1; Meilenstein “Tool interface
description evaluated”). In Leipzig wurden Beschreibungen einzelner hauseigener Tools in Galaxy
integriert (mea – ein Tool zur Vorhersage von maximum expected accuracy Strukturen von RNAs
und zum Vergleich von RNA-Strukturen; metilene – ein Tool zur differentiellen
Methylierungsanalyse). Während diese Tools direkt in Galaxy’s nativer Beschreibungssprache
(XML-basiert) erstellt wurden, haben wir darüber hinaus Galaxy-Interfacebeschreibungen
automatisiert aus der vorhandenen Interfacebeschreibung von Tools aus dem
LocARNA-Softwarepacket (z.B. locarna, sparse, exparna-p) erzeugt. Die daraus gewonnenen
Erfahrungen fliessen in die Entwicklung einer allgemeinen “Meta”-Interfacebeschreibungssprache
ein, aus der verschiedene Interfacebeschreibungen generiert werden können. Dies soll die
Galaxy-Toolintegration stark vereinfachen und die Nachhaltigkeit der Integrationsarbeit
sicherstellen (Workpackage 1.2, Meilenstein “Interface description implemented”, Meilenstein “First
set of tools integrated in Galaxy”).
Zu Workpackage 4.1 (Meilenstein “First quality control pipeline adapted”) wurden die spezifischen
Erfordernisse RNA-bezogener Experimente analysiert, um Standard Pipelines (z.b. FastQC)
anzupassen. Diese Analyse wird fortgeführt, da neue experimentelle Methoden zur
RNA-Interaktionsanalyse (z.B. ChIP-Seq, CLIP-Seq), die Adaptierung und Neuentwicklung von
QC-Methoden erfordern. Wir beginnen bereits etablierte Methoden anzupassen und in das Galaxy
Framework zu integrieren.
Zur Identifikation von Benutzeranforderungen in Workpackage 6.1, wurden regelmäßig auftretende
direkte Benutzeranfragen bearbeitet und analysiert. Für Workpackage 6.2 wurde die bestehenden
Software weiter gepflegt, an spezifische Benutzerbedürfnisse angepasst und das Deployment
verbessert (z.B. Debian/Ubuntu packages für LocARNA). An diesen Punkten wurde im
Berichtszeitraum kontinuierlich gearbeitet. Dies führte zur Entwicklung neuer Tools, die zur
Integration in das Framework der Galaxy RNA Workbench vorgesehen sind, und neuen Releases
schon vom RBC zur Verfügung gestellter Tools. Unter anderem wurde veröffentlicht (Meilenstein
“Scheduled software updates and newly released software of software portfolio”) :
Bezüglich Workpackage 8.1, koorgansierte Leipzig den RNA workshop für Mitglieder des SPP
1738 und organisierte den Workshop “Bio-visualisierung” auf einem gemeinsamen Seminar mit
Bioinformatikgruppen aus Wien und Kopenhagen (Meilenstein “Two training courses in RNA
bioinformatics”). Weiterhin wurde in direkter Kooperation Support zum Design von Riboswitches
(Prof. Moerl) und genomweitem Structureprobing (Prof. Narberhaus, Prof. Dersch) geleistet. Für
Workpackage 8.2 wurde die Dokumentation unserer Tools weitergepflegt. Insbesondere, wurde
LocARNA’s PP-format spezifiziert, um den Austausch zwischen verschiedenen Tools zu
vereinheitlichen und die Standardisierung als EDAM Format vorzubereiten.

----- Jun 2016
Im Rahmen von Workpackage 4.1 (Meilenstein “First quality control
pipeline adapted”) wurden die spezifischen Erfordernisse RNA-bezogener
Experimente und bestehende Tools zur Qualitaetskontrolle
erhoben. Diese Recherche wird fortgeführt, da neue experimentelle
Methoden zur RNA-Interaktionsanalyse (z.B. ChIP-Seq, CLIP-Seq), die
Adaptierung und Neuentwicklung von QC-Methoden erfordern. Bereits
etablierte Methoden werden angepasst und in das Galaxy Framework
integriert. Die Ergebnisse der Literaturrecherche werden in Form von
SOPs zusammengefasst und \"offentlich zur Verf\"ugung gestellt. Der
Aufbau eines zentralen Anlaufpunkts f\"ur RNA-Bioinformatik-bezogene
Dokumentation und Trainingsmaterial (SOPs, Howto-Dokumente, Tutorials)
ist in Planung.

Zur Identifikation von Benutzeranforderungen in Workpackage 6.1,
wurden weiterhin regelm\"assig auftretende direkte Benutzeranfragen
bearbeitet und analysiert. F\"ur Workpackage 6.2 wurde die bestehenden
Software weiter gepflegt, an spezische Benutzerbed\"urfnisse angepasst
und das Deployment weiter verbessert (z.B. Conda packages f\"ur
verschiedene Tools).

Weiterhin, wurden im Berichtszeitraum neue Software-releases
ver\"offentlicht (Meilenstein ``Scheduled software updates and newly
released software of software portfolio'') :


----Dez 2016
Die Erhebung der spezifischen Erfordernisse RNA-bezogener Experimente
und bestehender Tools zur Qualitätskontrolle im Rahmen von Workpackage
4.1 wird laufend fortgesetzt. Um mit der schnellen Entwicklung
RNA-bezogener Experimente und den sich daraus ergebenden Anforderungen
an die Qualitätskontrolle Schritt zu halten wurde eine RBC Kollektion
von How-Tos und Tutorials gestartet. Letztere agieren als öffentlich
erreichbarer, zentraler Anlaufpunkt für Forscher die an solchen
Themen interessiert sind. Diese Kollektion wurde in Kollaboration mit
Allen RBC Partnern als Alternative zu SOPs gewählt, da eine solche
Kollektion einfacher zu warten und auf den neuesten Stand zu bringen
ist und eine erhöhte Sichtbarkeit erhofft wird.  Die Anpassung an
neue experimentelle Methoden, z.B. zur RNA-Interaktionsanalyse
(e.g. CLIP-Seq, ChIP-Seq, PARIS, SHAPE) und sich daraus ergebenden
notwendigen Adaptierung von Analyse- und Qualitätskontrollschritten
wurde begonnen.
Als Grundlage für die Anpassung von Standard Qualitycontrolpipelines
(z.B. FastQC), wurden zun\"achst die spezifischen Erfordernisse
RNA-bezogener Experimente analysiert. 
Es wurden weiterhin regelm\"assig auftretende direkte Benutzeranfragen
bearbeitet und analysiert um Benutzeranfordungen gemäß Workpackage
6.1 zu identifizieren.  Bestehende Software wurde f\"ur Workpackage
6.2 weiter gepflegt, das Deployment über Conda packages weiter
forciert und spezifische Benutzerbed\"urfnisse eingearbeitet.

 Bez\"uglich Workpackages 7.3 und 8.1 wurde eine Summer
School zu RNA-Seq f\"ur SPP 1738 durchgeführt, die in Zusammenarbeit
mit dem Bioinformatik-Dienstleister ecSeq abgehalten wurde. Die
dar\"uberhinausgehende Kooperation mit ecSeq ist angebahnt. Die
Dokumentation unserer Tools wurde weitergepflegt; eine Reihe von Tools
wurde in die ELIXIR-Registry, die zentrale Katalogisierung und
standardisierte Beschreibung von Bioinformatik-Tools erm\"oglicht,
eingetragen und diese Registry in Kooperation mit ELIXIR
weiterentwickelt.
\end{comment}


\section*{Tools and services developed and provided}
\label{sec:tools}
Description and information on the usage of tools and services
Information on the development of tools and services 
Table: List of tools and services 
• Vienna RNA package 2.2.0-RC3 – RNA-Strukturvorhersage mit Softconstraints
• metilene 0.2-4 – differentielle Methylierungsanalyse
• LocARNA 1.8.7 – multiple structure-based RNA alignment and folding
• SPARSE 1.8.7 – highly efficient structure-based RNA alignment
• MEA 0.6.4 – maximum expected accuracy prediction
• SparseMFEFold 0.3 – space-efficient RNA folding• CARNA 1.3 – constraint-based alignment of RNAs
• ein neuer Variant-Caller (für Segemehl 0.2)
• RNAcop –Kontextoptimierung mit Wahrscheinlichkeiten
• Detektion von tRNA-remolding-Ereignissen
Zur Unterstützung der Toolintegration in Galaxy, wird schrittweise Continuous Integration Testing
aller Programme und Galaxy tool mit Travis eingeführt (Meilenstein “First set of tools integrated in
Galaxy”).

\begin{itemize}
\item Vienna RNA package 2.2.5 – RNA-Strukturvorhersage mit
  Softconstraints
\item metilene 0.2-6 – differentielle Methylierungsanalyse
\item LocARNA 1.8.9 – multiple structure-based RNA alignment and
  folding
\item SPARSE 1.8.9 – highly efficient structure-based RNA alignment
\item MEA 0.6.4 – maximum expected accuracy prediction
\item SparseMFEFold 0.4 – space-efficient RNA folding
\item CARNA 1.3.1 – constraint-based alignment of RNAs
\item ViennaNGS Bio::ViennaNGS 0.17\_01
\end{itemize}

F\"ur die Tools LocARNA, CARNA, und IntaRNA, Metilene, ViennaRNA,
ViennaNGS, und Segemehl wurden Packages für das von Galaxy genutzte
Paketmanagementsystem Conda erstellt (Meilenstein “First set of tools
integrated in Galaxy”), die auch allgemein das Deployment dieser Tools
stark vereinfachen. 

Im Berichtszeitraum wurden folgende neue Software-releases
ver\"offentlicht (Workpackage 6.1 und 6.3):

\begin{itemize}
\item Vienna RNA package 2.3.1 – RNA-Strukturvorhersage mit
  Softconstraints
\item LocARNA 1.8.12 – multiple structure-based RNA alignment and
  folding
\item SPARSE 1.8.12 – highly efficient structure-based RNA alignment
\item Sierra platinum - multi-replicate peak-calling and quality control of NGS data
\end{itemize}

Packages für das von Galaxy genutzte Paketmanagementsystem Conda
wurden für Workpackage 2.2 erstellt. Neu hinzugekommen sind Packages
fur das Tool mlocarna, bestehende packages für LocARNA, CARNA,
IntaRNA, Metilene, ViennaRNA und ViennaNGS, die auch allgemein das
Deployment dieser Tools stark vereinfachen, wurden an neue Version
angepasst.
Passend zu Workpackage 4.1 und
6.3 wurde Sierra platinum, ein Webserver für multi-replicate
peak-calling und quality control von NGS daten veröffentlicht und
derzeit in das Galaxy framework integriert.


\section*{Contribution to de.NBI activities}
\label{sec:activities}
Contributions to workshops; summer schools and symposia
Contributions to the de.NBI cloud  (if applicable)
Further plans
Bez\"uglich Workpackages 7.3 und 8.1 wurde eine
Summer School zu RNA-Seq f\"ur SPP 1738 geplant, die in Zusammenarbeit
mit dem Bioinformatik-Dienstleister ecSeq abgehalten wird. Die
dar\"uberhinausgehende Kooperation mit ecSeq ist angebahnt.
Bez\"uglich Workpackage 8.2 wurde die Dokumentation unserer Tools
weitergepflegt; insbesondere wurden eine Reihe von Tools in die
ELIXIR-Registry, die zentrale Katalogisierung und
standardisierte Beschreibung von Bioinformatik-Tools erm\"oglicht,
eingetragen.



\section*{Contributions to ELIXIR activities}
\label{sec:elixier}
Contributions to implementations projects and core data resources
Other contributions 

\section*{General information on the project}
\label{sec:general}
Composition of the project group 
Up to three important publications 

Prof. Dr. Peter F. Stadler is full professor of Bioinformatics at Leipzig University, External
Scientific Member at the MPI-MIS, where he directs a research group on discrete
biomathematics, senior scientific advisor to the RNomics group at FH IZI in Leipzig, and an
external professor at the Santa Fe Institute. He has pioneered RNA bioinformatics with the
ViennaRNA package since the early 1990s. Beyond core algorithms for RNA folding his
group has developed scanning algorithms for large genomes, facilities to compute
consensus structures, and methods to deal with RNA-RNA interactions. Complementary, the
group develops methods for the analysis of high throughput transcriptomics data. In
particular, focusing on read mapping, functional RNA recognition from short read patterns,
and detection of potential chemical modifications in RNA-seq data.
In addition to the applicant, the following researchers from AG Stadler are or have been 
part of the RBC-Leipzig. Until 11.2016, Dr. Sebastian Will, since 01.2016 Dr. Jörg Fallmann, since 11.2016 Dipl. Bioinf. Jan Engelhardt. Together they have strong experience with development and maintenance of several RNA related software suites and databases concerning structure prediction, interaction, folding dynamics, and comparative analysis of structural RNA, as well as genome-wide prediction of non-coding RNA (CARNA, LocARNA, MEA, AREsite, etc. )

\begin{itemize}
\item[{[P1]}] Washietl S, Hofacker IL, \textbf{Stadler, PF}. 
  Fast and reliable prediction of noncoding RNAs.
  \textit{Proc.\ Natl.\ Acad.\ Sci.\ USA} \textbf{102}: 2454-2459 (2005).
\item[{[P2]}] Washietl S, Hofacker IL, Lukasser M, H{\"u}ttenhofer A, 
  \textbf{Stadler, PF}. Mapping of conserved {RNA} Secondary Structures 
  predicts Thousands of functional Non-Coding RNAs in the Human Genome.
  \textit{Nature Biotech.} \textbf{23}: 1383-1390 (2005).
\item[{[P3]}] Washietl S, Pedersen JS, Korbel JO, Gruber A, Hackerm{\"u}ller 
  J, Hertel J, Lindemeyer M, Reiche K, Stocsits C, Tanzer A, Ucla C,
  Wyss C, Antonarakis SE, Denoeud F, Lagarde J, Drenkow J, Kapranov P,
  Gingeras TR, Guig{\'o} R, Snyder M, Gerstein MB, Reymond A, Hofacker IL,
  \textbf{Stadler PF}. Structured RNAs in the ENCODE Selected Regions
  of the Human Genome. \textit{Genome Res.} \textbf{17}: 852-864 (2007).
\item[{[P4]}] Rose D,  Hackerm\"{u}ller J, Washietl S, Findei{\ss} S, 
  Reiche K, Hertel J, \textbf{Stadler PF}, Prohaska, SJ.
  Computational {RNomics} of Drosophilids. \textit{BMC Genomics} \textbf{8}:
  406 (2007).
\item[{[P5]}] Gruber AR, Findei{\ss} S, Washietl S, Hofacker IL, 
  \textbf{Stadler PF}. \texttt{RNAz 2.0}: improved noncoding RNA detection.
  Pac.\ Symp.\ Biocomput. \textbf{15}: 69-79 (2010).
\item[{[P6]}] Findei{\ss} S, Engelhardt J, Prohaska SJ, \textbf{Stadler PF}.
  Protein-Coding Structured RNAs: A Computational Survey of Conserved RNA 
  Secondary Structures Overlapping Coding Regions in Drosophilids.
  \textit{Biochimie} \textbf{93}: 2019-2023 (2011).
\item[{[P7]}] Lorenz R, Bernhart SH, H{\"o}ner zu Siederdissen C, Tafer H,
  Flamm C, \textbf{Stadler PF}, Hofacker IL.  \texttt{ViennaRNA} Package
  2.0. \textit{Alg.\ Mol.\ Biol.} \textbf{6}: 26 (2011).
\item[{[P8]}] Lorenz R, Bernhart SH, Qin J, H{\"o}ner zu Siederdissen C, 
  Tanzer A, Amman F, Hofacker IL, \textbf{Stadler PF}. 
  2D meets 4G: G-Quadruplexes in RNA Secondary Structure Prediction.
  \textit{IEEE Trans. Comp. Biol. Bioinf.} \textbf{10}: 832-844 (2013).
\item[{[P9]}] Smith MA, Gesell T, \textbf{Stadler PF}, Mattick, JS.
  Widespread purifying selection on RNA structure in mammals.
  \textit{Nucleic Acids Res.} \textbf{41}: 8220-8236 (2013).
\item[{[P10]}] Lorenz R, Hofacker IL, \textbf{Stadler PF}. RNA Folding with
  Hard and Soft Constraints. \textit{Alg. Mol. Biol.} \textbf{11}: 8 (2016).
\end{itemize}

\end{document}
